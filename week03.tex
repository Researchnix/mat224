\documentclass[letterpaper, 10pt]{article}
\usepackage[margin=1in]{geometry} 
\usepackage{amsmath,amsthm,amssymb,scrextend}
\usepackage{fancyhdr}
\pagestyle{fancy}
\usepackage{silence}
\WarningFilter{latex}{You have requested package}
\input{ltx/pkg/preamble}





\begin{document}

\lhead{MAT224 Linear Algebra II}
\chead{Bases, Dimension \& Linear Transformations}
\rhead{Week 03}

\title{Linear Algebra II \\ \Large{MAT224}}
\author{Lennart Döppenschmitt}
% \maketitle
% \tableofcontents

\section*{Bases and Dimension}%
\label{sec:title}

\textbf{Textbook:} Section 1.6


\lb
\textbf{Definition 1.6.1}
\lb
A family of vectors $\cal B$ in a vector space $V$ is called \emph{a basis of $V$} if
\begin{enumerate}
    \item $\cal B$ spans $V$
    \item $\cal B$ is linearly independent
\end{enumerate}



\lb
\textbf{Examples} 
\begin{enumerate}
    \item $ \cb{\vec e_1, \ldots, \vec e_n} $ is a basis of $\R^n$.
    \item We have seen last week that $ \cb{\ttpl{1}{1}, \ttpl{-1}{1}} $ is a basis of $\R^2$
    \item Which of families of polynomials we have seen before is a basis of $\pol{2}{\R}$?
        Can you write down a basis of $\pol{2}{\R}$ that does not contain any monomials?
\end{enumerate}



\lb
\textbf{Theorem 1.6.3}
\lb
A family of vectors $\cal B$ is a basis of $V$ if and only if every vector
$\vec v ∈ V$ can be written uniquely as a linear combination of vectors in
$\cal B$.

\begin{proof}
\end{proof}



\lb
\textbf{Remark} 
\begin{enumerate}
    \item A vector space does not have just one unique
        basis as we can easily verify.
\end{enumerate}


\lb
\textbf{Theorem 1.6.6} 
\lb
Let $V$ be a vector space with a finite spanning set.
For every linearly independent family $S$ in $V$, there
is a basis $\cal B$ containing $S$.

\lb
Why do we need a finite spanning set for $V$? Some vector spaces, such as
$\pol{}{\R}$ can not be spanned by finitely many polynomials.
Can you show why?


\lb
Observe that a basis hits a sweet spot of a family that is not too
large as that it would contain redundant vectors, but also not too
small of a family that it couldn't span the vetcor space.
\pr
The \emph{Extend} and \emph{Reduce} theorems from last week
give us the following:
\lb
A family of vectors that is linearly independent, but not a spanning set can
be enlagred to a basis, a family that spans $V$, but is linearly
dependent, can be cut down to form a basis.


\newpage
\lb
Even though a basis is not unique to a vector space, we would like to extract
an invariant, a label, something that characterizes the vectors space.
This invariant is motivated by the Corollary following below.

\lb
\textbf{Theorem 1.6.10} 
\lb
If $V$ is spanned by a family $S$ with $m$ elements, then no linearly
independent family $R$ in $V$ can have more than $m$ elements.
\begin{proof}
    
\end{proof}


\lb
\textbf{Corollary 1.6.11}
\lb
Any two bases $\cal B$ and $\cal B'$ of $V$ have the same number of elements



\lb
\textbf{Definitions} 
\begin{enumerate}
    \item If a vector space $V$ has a finite basis, we say that $V$ is
        \emph{finite dimensional}.
    \item For a finite dimensional vector space $V$, the \emph{dimension}
        of $V$
        \[ \dim(V) \]
        is the number of elements of a basis of $V$.
\end{enumerate}



\lb
\textbf{Discussion} 
\lb
What is the dimension of 
\begin{enumerate}
    \item[] $\dim( \R ) = $
    \item[] $\dim( \pol{n}{\R} ) = $
    \item[] $\dim( \mat{2}{\R} ) = $
    \item[] $\dim( \smat{2}{\R} ) = $
    \item[] $\dim( \amat{2}{\R} ) = $
\end{enumerate}
\lb
Remember that a matrix $A$ is symmetric if $\tp A = A$ and antisymmetric if $\tp A = -A$.



\newpage
\lb
\textbf{Discussion}
\lb
Can you argue that if $U \subseteq V$ is a subspace then $\dim (U) \leq \dim(V)$?
\pr
Is the converse true that if $\dim (U) = \dim(V)$ then $U = V$?






































\end{document}
