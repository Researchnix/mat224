\documentclass[letterpaper, 10pt]{article}
\usepackage[margin=1in]{geometry} 
\usepackage{amsmath,amsthm,amssymb,scrextend}
\usepackage{fancyhdr}
\pagestyle{fancy}
\usepackage{silence}
\WarningFilter{latex}{You have requested package}
\input{ltx/pkg/preamble}





\begin{document}

\lhead{MAT224 Linear Algebra II}
\chead{Kernel, Image \& Dimension Theorem}
\rhead{Week 05}

\title{Linear Algebra II \\ \Large{MAT224}}
\author{Lennart Döppenschmitt}
% \maketitle
% \tableofcontents



\lb
\textbf{Announcements for the test} 
\begin{enumerate}
    \item Don't quote numbers of Theorems, but their content and why they apply in this case.
    \item Famous theorems have names, for example \emph{Extend-,} \emph{Reduce-} and \emph{Fundamental} Theorem.

    \item The test covers everything we did from sections 1.1 - 1.6 and section 2.1,
        homework problems 1 to 4 and Assignments 1 \& 2

    \item Please read the \emph{Term-Test-1 information document} about logistics
        well in advance!
    \item \emph{Don't} leave the upload to the last minute!
        Plan for some technical difficulties, they always occur! :(

    \item
        Read question statements carefully and slowly! Make sure you distinguish \emph{if} vs.
        \emph{if and only if} in statements. Give a proof or explanation when the question
        requires one.
    \item
        While we write $\vec v$ for vectors, the book and other sources may write
        $\textbf{x} ∈ V$.
    \item
        By abuse of notation, the image of a vector under a transformation $T(\vec v)$ can
        sometimes be abbreviated as $T \textbf{x} $.

\end{enumerate}



\vspace{100pt}
\section*{Kernel and Image}%
\textbf{Textbook:} Section 2.3


\lb
\textbf{Definition (2.3.1 \& 2.3.10)}
\lb
For a linear transformation $ \map{V}[T]{W}$, we define
\begin{enumerate}
    \item
        the \emph{preimage} $T^{-1}(S)$ of $S \subseteq W$ under $T$ as all $\vec v ∈ V$ that map into $S$.
    \item
        the \emph{kernel} $\ker(T)$ of $T$  as all $\vec v ∈ V$ that map to $\vec 0$ under $T$,
    \item 
        the \emph{image} $\tx{im}(T)$ of $T$ as all $\vec w ∈ W$ such that $\vec w = T(\vec v)$ for some $\vec v ∈ V$,
\end{enumerate}




\vspace{100pt}
\lb
\textbf{Example}
\begin{itemize}
    \item
    The kernel of $\map{\pol{n}{\R}}[\frac{d}{dx}]{\pol{n}{\R}}$ are all constant polynomials,
    while the image consists of polynomials of degree $n-1$.

    \item
    The kernel of the evaluation map $\map{\pol{n}{\R}}[\tx{ev}_2]{\R}$ are all
    polynomials that have a root at $x = 2$. What is the image?

    \item
    What is the image of the linear transformation defined in example 2.2.2 $\map{\R^2}[T]{\R^2}$?
    \[ T(\vec e_1) = \vec e_1 + \vec e_2 \]
    \[ T(\vec e_2) = \vec 2 e_1 - 2 \vec e_2 \]
\end{itemize}


\newpage
\lb
\textbf{Proposition 2.3.2 \& 2.3.11}
\lb
For every linear transformation $\map{V}[T]{W}$
\begin{enumerate}
    \item $\ker(T)$ is a subspace in $V$
    \item $\tx{im}(T)$ is a subspace in $W$.
\end{enumerate}
\begin{proof}
\end{proof}







\vspace{300pt}
\lb
\textbf{Proposition 2.3.7}
\lb
Let $\map{V}[T]{W}$ be a linear transformation between vector spaces $V$ and $W$ with
bases $α$ and $β$ respectively.
\lb
\ble{
    The subspace $\ker(T)$ is isomorphic to the solution space to
    the homogeneous system of $[T]_α^β$.
}
\lb
\q{
    That is,
    \begin{itemize}
        \item
            for every $\vec v ∈ \ker(T)$ the coordinate tuple $[\vec v]_α$
            solves $[T]_α^β \vec x = \vec 0$.
        \item
            for every solution $ \vec s = \begin{pmatrix} s_1 \\ \vdots \\ s_n \end{pmatrix}$
            of the system $[T]_α^β \vec x = \vec 0$, the \emph{realization} in the basis $α$, namely
            \[ \iv{(γ^α)} \left(\begin{pmatrix} s_1 \\ \vdots \\ s_n \end{pmatrix} \right) \]
            is a vector in $\ker(T)$.
    \end{itemize}
}
\begin{proof}
\end{proof}




\newpage
\lb
\textbf{Example}
\lb
Find the kernel of the evaluation map $ \map{\pol{n}{\R}}[\tx{ev}_2]{\R}$.




\newpage
\lb
\textbf{Observation (analogous to 2.3.7)}
\lb
Let $\map{V}[T]{W}$ be a linear transformation between vector spaces $V$ and $W$ with
bases $α$ and $β$ respectively.
\lb
\ble{
    The subspace $\tx{im}(T)$ is isomorphic to the space of all $\vec b ∈ \R^m$ such that
    the system of linear equations $[T]_α^β \vec x = \vec b$ has a solution
}
\lb
\q{
    That is,
    \begin{itemize}
        \item
            for every $\vec w ∈ \tx{im}(T)$ the system
            $[T]_α^β \vec x = [\vec w]_β$ has a solution.
        \item
            for every constant vector $ \vec b = \begin{pmatrix} b_1 \\ \vdots \\ b_n \end{pmatrix}$
            such that the system $[T]_α^β \vec x = \vec b$ has a solution, the \emph{realization} in the basis $β$, namely
            \[ \iv{(γ^β)} \left(\begin{pmatrix} b_1 \\ \vdots \\ b_n \end{pmatrix} \right) \]
            is a vector in $\tx{im}(T)$.
    \end{itemize}
}
\begin{proof}
\end{proof}


\vspace{200pt}

\lb
\textbf{Discussion} 
\lb
Based on this observation, which $\vec v ∈ V$ will be in the preimage $T^{-1}(\vec w)$?









\newpage
\lb
\textbf{Definition}
\lb
For a matrix $A = [ a_1, a_2, \ldots, a_m] ∈ \mat{n, m}{\R}$ we denote
the span of the columns of $A$ by
\[ \tx{col}(A) = \spn{a_1, \ldots, a_m} \]


\vspace{100pt}
\lb
\textbf{Lemma}
\lb
For a matrix $A = [ a_1, a_2, \ldots, a_m] ∈ \mat{n, m}{\R}$ the system
\[ A\vec x = \vec b \]
has a solution if and only if
$\vec b ∈ \tx{col}(A)$
\begin{proof}
\end{proof}


\vspace{100pt}
\lb
\textbf{Proposition}
\lb
Let $\map{V}[T]{W}$ be a linear transformation between vector spaces $V$ and $W$ with
bases $α$ and $β$ respectively.
\lb
The image of $T$ is isomorphic to $\tx{col}([T]_α^β)$ under the coordinate map
$ \map{W}[γ^β]{\R^m}$.
\begin{proof}
\end{proof}

\vspace{100pt}
\lb
\textbf{Example}
\lb
Find the image of the linear transformation 
\begin{align*}
    \R^2 &\ra \R^3 \\
    \ttpl{x}{y} &\mapsto \tttpl{x+y}{x}{y}
\end{align*}




\newpage
\lb
Notice that the columns might not be independent,
in which case the columns are a spanning set of the image, but not a basis.


\lb
\textbf{Theorem (Nicholson - Linear algebra with Applications)}
\lb
Given a linear transformation $\map{V}[T]{W}$ with matrix $[T]_α^β$ for some bases $α$ and $β$.
Let $R = \tx{RREF}([T]_α^β)$ be the reduced row echolon form of $[T]_α^β$.
\lb
Then if the leading 1s in are $R$ lie in columns
$j_1, j_2,\ldots, j_r$, the columns $j_1, j_2,\ldots, j_r$ of $[T]_α^β$ are a basis for 
$ \tx{col}([T]_α^β)$
\begin{proof}
\end{proof}


\vspace{200pt}
\lb
\textbf{Discussion} 
\lb
Suppose a linear transformation $\map{V}[T]{W}$ is given in a some bases $α$ and $β$ by
\[ [T]_α^β = \begin{pmatrix} 1 & 2 & 0 & 1 \\ 1 & 2 & 1 & 0 \end{pmatrix} \]
Find a basis for $\tx{im}(T)$ and $\ker(T)$.




% \lb
% \textbf{Proposition 2.3.12}
% \lb
% If $ \cb{\vec v_1, \ldots, \vec v_k}$ spans $V$, then $ \cb{T(\vec v_1), \ldots, T(\vec v_k)}$
% spans $\tx{im}(T)$.
% \begin{proof}
% \end{proof}



\newpage
\lb
A similar question is discussed in Example 2.3.9 in the book.


\lb
\textbf{Discussion}
\lb
Find the kernel, the image and bases of them for
the following transformations in bases of your choice.

\begin{enumerate}
    \item tbd
\end{enumerate}






\newpage
\section*{Dimension Theorem}%

\textbf{Textbook:} Section 2.3 \& 2.4
\lb
\textbf{Theorem 2.3.17 (Rank-Nullity)}
\lb
For any linear transformation $\map{V}[T]{W}$
\[ \dim(V) = \dim(\ker(T)) + \dim( \tx{im}(T)) \]


\lb
\textbf{Remark} 
\begin{itemize}
    \item 
        $\dim( \tx{im}(T))$ is the same as the rank of $[T]_α^β$ and by abuse of
        notation also referred to as $\tx{rank}(T)$.
    \item
        Some books refer to $\dim(\ker(T))$ as the \emph{nullity} of $T$.
\end{itemize}

\begin{proof}
\end{proof}


\vspace{300pt}
\lb
\textbf{Theorem}
\lb
A linear transformation $T$ is injective if and only if $\ker(T) = \cb{\vec 0}$
\begin{proof}
\end{proof}



\newpage
\lb
\textbf{True or False} 
Let $\map{V}[T]{W}$ be a linear transformation
\begin{enumerate}
    \item[$\square$]
        If $T$ is an isomorphism, then $\dim(V) = \dim(W)$.
    \item[$\square$]
        If $\dim(V) > \dim(W)$, $T$ has to be injective.
    \item[$\square$]
\end{enumerate}







\newpage
\section*{Composition}%
\textbf{Textbook:} Section 2.5



\lb
\textbf{Discussion}
\lb
Without doing a lot of work, can you argue what the matrix
representing the composition $F \circ T$ is assuming you know $[F]$ and $[T]$?








\newpage
\section*{Review session}%
\begin{itemize}
    \item Vector spaces
    \item subspaces
    \item Sum, intersection, direct sum
    \item Linear independence
    \item Spanning set
    \item Basis
    \item Dimension
    \item Linear Transformation
\end{itemize}




\end{document}
