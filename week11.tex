\documentclass[letterpaper, 10pt]{article}
\usepackage[margin=1in]{geometry} 
\usepackage{amsmath,amsthm,amssymb,scrextend}
\usepackage{fancyhdr}
\pagestyle{fancy}
\usepackage{silence}
\WarningFilter{latex}{You have requested package}
\input{ltx/pkg/preamble}





\begin{document}

\lhead{MAT224 Linear Algebra II}
\chead{Jordan Canonical Form}
\rhead{Week 11}

\title{Linear Algebra II \\ \Large{MAT224}}
\author{Lennart Döppenschmitt}
% \maketitle
% \tableofcontents

\section*{Jordan Canonical Form}%
\textbf{Textbook:} Section 6.3 \& 6.4

\lb
\textbf{Goal:} We Would would like to improve the upper triangular form of
triangulizable matrices.


\lb
\textbf{Discussion}
\lb
Suppose we have the linear transformation $T(p) = \frac{d^2}{dx^2} p + p$ on $\pol{3}{\bb C}$.
\begin{enumerate}
    \item What are the eigenvalues of $T$?
    \item Why is $N = T - I$ nilpotent?
    \item What is the nilpotent Jordan form of $N$?
    \item Find a canonical basis $α$ of $N$, what is $[T]_α^α$ in this basis?
\end{enumerate}

\newpage
\lb
\textbf{Notation}
\lb
\begin{enumerate}
    \item \emph{Jordan block} of size $k \times k$ $J_m(λ)$
    \vspace{140pt}
\item \emph{Jordan matrix} $J = J_{m_1}(λ_1) \oplus \cdots \oplus J_{m_k}(λ_k)$
\end{enumerate}



\vspace{200pt}
\lb
\textbf{Discussion}
\lb
Let $T$ be a linear transformation on a complex vector space of dimension $4$ with only a single
eigenvalue $λ_1$.
\begin{enumerate}
    \item What is the characteristic polynomial $c_T(λ)$?
    \item Is $T- λ_1 I$ nilpotent?
    \item List all possible Jordan forms of $T$.
\end{enumerate}
\textbf{Hint: } Corollary (6.1.11) says that a triangulizable matrix can in particular be written
in a canonical form where the diagonal only contains eigenvalues.



\newpage
\lb
\q{\textbf{Problem: } What if $T$ on $V$ has more than one distinct eigenvalue? }

\lb
\textbf{Notation}
\lb
\begin{enumerate}
    \item Let $T$ be a linear transformation on $V$ with an invariant subspace $W \subseteq V$.
        We denote by $T \vert _W$ the restriction of $T$ to $W$.
    \item Suppose $V = W_1 \oplus W_2$ is the direct sum of two invariant subspaces of $T$.
        $T$ is fully determined by its restrictions to $W_1$ and $W_2$.
        Moreover, if $α$ and $β$ are bases of $W_1$ and $W_2$ respectively
        \[ [T]_γ^γ = [T]_α^α \oplus [T]_β^β \]
        in the basis $γ = α \cup β$ as a direct sum of matrices.
\end{enumerate}



\vspace{100pt}
\lb
\textbf{Goal}
\lb
Why does this help us? If we can find invariant subspaces $W_1, \ldots, W_k$ such that
$T \vert _{W_i}$ has only one distinct eigenvalue $λ_i$, we can find the Jordan canonical form
of $T \vert _{W_i}$ and take the direct sum for each invarinat subspace.



\vspace{100pt}
\lb
\textbf{Definition (6.3.2)}
\lb
Let $T$ be a linear transformation on a finite-dimensional vector space $V$ with an eigenvalue
$λ$ of multiplicity $m$
\begin{enumerate}
    \item The \emph{λ-generalized eigenspace} $K_λ$ is teh kernel of the transformation
        $(T-λI)^m$ on $V$.
    \item The nonzero elements of $K_λ$ are called \emph{generalized eigenvectors} of $T$
        with eigenvalue $λ$.
\end{enumerate}



\newpage
\lb
\textbf{Discussion}
\lb
Let $V$ be a vector space with basis $α = \cb{\vec v_1, \ldots, \vec v_5}$ and $T$ a linear
transformation on $V$ such that
\[ [T]_α^α = J_2(2) \oplus J_2(3) \oplus J_1(3) \]
\begin{enumerate}
    \item Find all eigenvalues of $T$ with their multiplicity.
    \item Find all the eigenspaces of $T$.
    \item Find all of the generalized eigenspaces of $T$.
    \item Is $T$ diagonalizable?
\end{enumerate}



\newpage
\lb
\textbf{Discussion}
\lb
Let $V$ be a vector space with basis $α = \cb{\vec v_1, \ldots, \vec v_5}$ and $T$ a linear
transformation on $V$ such that
\[ [T]_α^α = J_2(2) \oplus J_2(3) \oplus J_1(3) \]
\begin{enumerate}
    \item Are the generalized eigenspaces invariant?
    \item Can you write $V$ as a direct sum of subspaces such that $T$ restricted to each
        of them has exactly one eigenvalue?
\end{enumerate}



\newpage
\lb
\textbf{Discussion}
\lb
Let $T$ be a linear transformation on a $V$ be a finite dimensional vector space with
eigenvalue $λ$ of multiplicity $m$.
\begin{enumerate}
    \item Show that $T \vert _{K_λ}$ has only the eigenvalue $λ$
    \item Is $(T-λI) \vert _{K_λ}$ nilpotent?
    \item Show that $K_λ$ is an invariant subspace in $V$.
\end{enumerate}


\newpage
\lb
\textbf{Proposition (6.3.4)}
\lb
Let $T$ be a linear transformation on a $V$ be a finite dimensional vector space.
\begin{enumerate}
    \item For each eigenvalue $λ$ of $T$, $K_λ$ is an invariant subspace of $V$.
    \item If $λ_1, \ldots, λ_k$ are all the distinct eigenvalues of $T$, then 
        $V = K_{λ_1} \oplus \cdots \oplus K_{λ_k}$.
    \item If $λ$ is an eigenvalue of multiplicity $m$, then $\dim(K_λ) = m$.
\end{enumerate}
\begin{proof}
\end{proof}



% another discussion from page 5/8


\newpage
\lb
\textbf{Theorem (6.3.6)} (Jordan Canonical Form)
\lb
Let $T$ be a linear transformation on a finite dimensinal vector space $V$ whose characteristic
polynomial has $\dim(V)$ roots in the field $F$ over which $V$ is defined. Then
\begin{itemize}
    \item $V$ has a \emph{canonical basis} $γ$ in which $[T]_γ^γ$ is a Jordan matrix.
    \item Moreover, this decomposition of $[T]_γ^γ$ into Jordan blocks is unique up to
        reordering of the Jordan blocks.
\end{itemize}

\lb
We call this the \emph{Jordan canonical form} of a linear transformation.
\lb
\begin{proof}
\end{proof}


\vspace{200pt}
\lb
\textbf{Discussion}
\lb
Let $T$ be a linear transformation on a vector space $V$ such that
\begin{enumerate}
    \item $c_T(λ) = (λ-2)^3 (λ+1)^4(λ-5)$
    \item The dimension of the eigenspaces $E_2, E_{-1}$ and $E_5$ are 1, 2 and 1, respectively.
    \item $(T+I)^2 \vert_{K_{-1}} = 0$
\end{enumerate}
Find the Jordan canonical form of $T$ with the eigenvalues listed in the order 2, $-1$, 5.




\newpage
\lb
\textbf{Discussion}
\lb
All matrices $A$ in $\mat{2}{\bb C}$ are similar to either
$ \begin{pmatrix} λ_1 & 0 \\ 0 & λ_2 \end{pmatrix} $ or
$ \begin{pmatrix} λ & 1 \\ 0 & λ \end{pmatrix} $.


\vspace{300pt}
\lb
In general, we can prove
\lb
\textbf{Corollary}
\lb
Two matrices are similar if and only if they have the same Jordan
canonical form up to reordering.

\newpage
\lb
\textbf{Exercise 2 in section 6.4}
\lb

% Which of these matrices are similar...



\end{document}
