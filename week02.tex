\documentclass[letterpaper, 10pt]{article}
\usepackage[margin=1in]{geometry} 
\usepackage{amsmath,amsthm,amssymb,scrextend}
\usepackage{fancyhdr}
\pagestyle{fancy}
\usepackage{silence}
\WarningFilter{latex}{You have requested package}
\input{ltx/pkg/preamble}





\begin{document}

\lhead{MAT224 Linear Algebra II}
\chead{Linear Combinations \& Independence}
\rhead{Week 02}

\title{Linear Algebra II \\ \Large{MAT224}}
\author{Lennart Döppenschmitt}
% \maketitle
% \tableofcontents

\section*{Linear combinations}%
\label{sec:Linear combinations}

\textbf{Textbook:} Section 1.3


\lb
\textbf{Definition 1.3.1}
\lb
\begin{enumerate}
\item
Let $V$ be a vector space and $ \cb{\vec v_1, \ldots, \vec v_k}$ be a collection of vectors
in $V$. A \emph{linear combination} of vectors $\vec v_1, \ldots, \vec v_k$ in $S$
with \emph{coefficients} $α_1, \ldots, α_k ∈ \R$ is a sum
\[ α_1 \vec v_1 + α_2 \vec v_2 +  \cdots + α_k \vec v_k \]
\item
A linear combination where all coefficients are zero is called \emph{trivial}.
\item
To abbreviate a linear combination as above, we may write
\[ \Sum[i=1][k] α_i \vec v_i\]
\end{enumerate}

\lb
Notice that this is the most arbitrary vector we can build with the vectors
$\vec v_1, \ldots \vec v_k$ available in $S$.


\lb
\textbf{Example}
\lb
\begin{enumerate}
    \item
        Polynomials in $\pol{3}{\R}$ are linear combinations of the \emph{monomials}
        $\cb{1, x, x^2}$.
    \vspace{200pt}
    \item
        Let $\vec e_1 = \tttpl{1}{0}{0} \vec e_2 = \tttpl{0}{1}{0}, \vec e_3 = \tttpl{0}{0}{1}$
        be the standard vectors in $\R^3$. Observe that every posible vector in
        $\R^3$ of the form $\tttpl{x}{y}{z}$ is a linear combination of these
        standard vectors.
\end{enumerate}




\newpage
% \lb
% \textbf{Lemma}
% \lb
% Let $V$ be a vector space with a subspace $U \subseteq V$ and
% $S = \cb{ \vec v_1, \ldots, \vec v_k}$ a collection of vectors.
% \lb
% If $\vec v_i ∈ U$ for all $i = 1, \ldots, k$, then all linear combinations are in $U$ as well.



\lb
\textbf{Definition 1.3.1} (continued)
\lb
Let $V$ be a vector space and $S = \cb{\vec v_1, \ldots, \vec v_k}$ be a collection of vectors
in $V$. The set of all linear combinations of vectors in $S$ is called the \emph{span of S}
and denoted
\[ \spn{S} = \spn{\vec v_1, \ldots, \vec v_k} \]

\pr
By convention, $\spn{\emptyset} = \cb{\vec 0}$.


\lb
\textbf{Example}
\lb
\begin{enumerate}
    \item \[ \pol{2}{\R} = \spn{1, x, x^2} \]
    \item \[ \mat{2}{\R} = \spn{\qquad ? \qquad } \]
\end{enumerate}


\newpage
\lb
\textbf{Discussion}
\lb
Let $R = \cb{\vec v_1, \ldots, \vec v_k}$ and
$S = \cb{\vec v_1, \ldots, \vec v_k, \vec v_{k+1}, \ldots, \vec v_{k+m}}$ be two collections
of vectors in a vector space $V$ such that $R$ is contained in $S$.
Show that $\spn{R}$ is contained in $\spn{S}$.


\vspace{300pt}


\lb
\textbf{Theorem 1.3.4}
\lb
Show that for any vector space $V$ and collection of vectors
$S = \cb{\vec v_1, \ldots, \vec v_k}$ in $V$
\[ \spn{S} \subseteq V\] is a subspace of $V$.
\begin{proof}
    
\end{proof}











\newpage
\lb
\textbf{Discussion}
\lb
Does the set of polynomials $ \cb{1-2x^2~,~ x^2 + x ~,~ x^3 - 3x^2 ~,~ 1}$ span $\pol{3}{\R}$?





\newpage
\section*{Linear Independence}%
\label{sec:Linear Independence}

\textbf{Textbook:} Section 1.4

\lb
\textbf{Prelude}
\lb
We have seen that a vector space can have different spanning sets, for example:
\begin{align*}
    \R^2 &= \tx{span} \Bigg \{ \ttpl{1}{0}, \ttpl{0}{1} \Bigg \} \\
    \R^2 &= \tx{span} \Bigg \{ \ttpl{1}{1}, \ttpl{-1}{1}, \ttpl{2}{0} \Bigg \}
\end{align*}
\q{How do these two spans differ?}
\lb
\lb
\lb
\lb
\lb
\lb
\lb
\textbf{Definition 1.4.2}
\lb
Let $S = \cb{\vec v_1, \ldots, \vec v_k} $ be a collection of vectors in a vector space $V$.
\begin{enumerate}
    \item 
        The collection of vectors is called \emph{linearly independent} if only the
        trivial linear combination of the vectors in $S$ is equal to zero.
        \pr
        That is,
        \[ α_1 \vec v_1 + \cdots + α_k \vec v_k = \vec 0 \]
        only for the coefficients $α_1 = α_2 = \cdots = α_k = 0$.
    \item
        In he opposite case, when there does exist a nontrivial combination of the vectors in $S$
        which is zero, we call the collection \emph{linearly dependent}.
\end{enumerate}


\lb
\textbf{Intuition}
\lb
When does a linear combinatin of vectors equal to zero? It means that the concatenation of
`arrows` representing the vectors results in a loop.
\lb
Moreover, such a loop is trivial if it is just a point, i.e. no interior area.



\newpage

\lb
\textbf{Example}
\lb
\begin{enumerate}
    \item The collection of polynomials $ \cb{1+x ~,~ 1-x ~,~ 1+2x}$ in $\pol{1}{\R}$ is linearly
        dependent
    \item The collection of monomials $ \cb{1 ~,~ x ~,~ x^2}$ in $\pol{2}{\R}$ is linearly
        independet
\end{enumerate}


\lb
\textbf{Discussion}
\lb
Decide if the following collections of vectors is linearly independent or dependent.
\begin{enumerate}
    \item todo
    \item todo
    \item todo
\end{enumerate}



\lb
\lb
\lb
\textbf{Lemma}
\lb
If $ \cb{\vec v_1, \ldots, \vec v_k}$ is a linearly dependent collection of vectors
in a vector space $V$, there exists an index $j ∈ \cb{1, \ldots, k}$ such that
\begin{enumerate}
    \item
        $\vec v_j ∈ \spn{\vec v_1, \ldots, \vec v_{j-1}}$
    \item
        $\spn{\vec v_1, \ldots, \widehat{\vec v_j}, \ldots, \vec v_k}
        = \spn{\vec v_1, \ldots, \vec v_k}$
\end{enumerate}
\begin{proof}
\end{proof}



\newpage
\lb
\textbf{Example}
\lb
More examples?!


\lb
\textbf{Theorem}
\lb
Let $ S = \cb{\vec v_1, \ldots, \vec v_k}$ be a collection of vectors in a vector space $V$.
\pr
Then $S$ is linearly independent if and only if every vector in $\spn{S}$ has a
unique representation as a linear combination.

\lb
That is, if and only if $S$ is linearly independent, we have that
\[ α_1 \vec v_1 + \cdots + α_k \vec v_k = β_1 \vec v_1 + \cdots + β_k \vec v_k \]
is equivalent with $α_i = β_i$ for all $i ∈ \cb{1, \ldots, k}$.
\begin{proof}
\end{proof}




\vspace{230pt}
\lb
\textbf{Theorem} (Extend \& Reduce)
\lb
Given ...




\newpage
\lb
\textbf{True or False?} (Theorems from above might be helpful)
\lb
\begin{enumerate}
    \item[$\square$]
        If $S$ is an independent collection of vectors and $R \subseteq S$, then
        $R$ is also independent.
    \item[$\square$]

\end{enumerate}



\end{document}
