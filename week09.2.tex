\documentclass[letterpaper, 10pt]{article}
\usepackage[margin=1in]{geometry} 
\usepackage{amsmath,amsthm,amssymb,scrextend}
\usepackage{fancyhdr}
\pagestyle{fancy}
\usepackage{silence}
\WarningFilter{latex}{You have requested package}
\input{ltx/pkg/preamble}





\begin{document}

\lhead{MAT224 Linear Algebra II}
\chead{Complex numbers and Fields}
\rhead{Week 09 II}

\title{Linear Algebra II \\ \Large{MAT224}}
\author{Lennart Döppenschmitt}
% \maketitle
% \tableofcontents

\section*{Complex numbers}%
\textbf{Textbook:} Section 5.1



\lb
\textbf{Motivation}
\lb
We have seen that the matrix $ A = \begin{pmatrix} 0 & -1 \\ 1 & 0\end{pmatrix}$ has the
characteristic polynomial $c_A(λ) = λ^2 + 1$ which has no real roots.
A root would be a square root $\sqrt{-1}$. Complex numbers solve this problem.

\lb
\textbf{Definition}
\lb
The set of complex number $\bb C$ is the set vector space $\R^2$ with an additional
multiplication of two complex numbers
\[ \ttpl{a}{b} \ast \ttpl{c}{d} = \ttpl{ac-bd}{ad+bc} \]

\lb
\textbf{Remark}
\lb
One often writes these coordinate tuples as vectors in the basis $ \cb{1, i}$.
The coordinate vector $\ttpl{a}{b}$ corresponds then to $a + i b$.


\lb
\textbf{Discussion}
\lb
\begin{enumerate}
    \item Express $\ttpl{2}{3}$ and $\ttpl{-1}{4}$ in the basis and determine
        $\ttpl{2}{3} \ast \ttpl{-1}{4}$
    \item Let $W = \spn{1} \subseteq \bb C$. Show that $W$ is closed
        under the multiplication $\ast$.
    \item Can we say that $W$ is isomorphic to $\R$? If so, why?
\end{enumerate}





\newpage
\lb
\textbf{Definition}
\lb
Let $z = a + ib ∈ \bb C$. We call $Re(z) = a$ the \emph{real part} of $z$ and $Im(z) = b$
the \emph{imaginary part} of $z$.



\lb
\textbf{Discussion}
\lb
Let $p(x) = x^2 + 1$ and $q(x) = x^4 + 1$.
\begin{enumerate}
    \item Find all roots of $p(x)$ in $\bb C$.
    \item Find all solutions to the equation $i \cdot z = 1$. Can you now give meaning to
        the expression $\iv i$?
    \item Find all roots of $q(x)$ in $\bb C$.
    \item What is the inverse of a complex number $a + ib$ in general?
\end{enumerate}


\newpage
\lb
\textbf{Discussion}
\lb
Let $w = a+ ib$ and consider $\map{\bb C}[T_w]{\bb C}$ to be $T_w(z) = w \cdot z$.
\begin{enumerate}
    \item Is $T_w$ a linear transformation between real vector spaces? If it is, what is the
        matrix representing it?
    \item Under what condition is $T_w$ invertible?
\end{enumerate}


\newpage
\lb
\textbf{Discussion}
\lb
Let  $\map{\bb C}[T]{\bb C}$ to be the function $T(a + ib) = a - ib$.
\begin{enumerate}
    \item Is $T_w$ a linear transformation between real vector spaces? If it is, what is the
        matrix representing it?
    \item What are eigenvalues and eigenspaces of $T$?
    \item What is $T \circ T$? Is $T$ invertible?
\end{enumerate}














\newpage
\section*{Fields}%
\textbf{Textbook:} Section 5.1

\lb
\textbf{Goal}
\lb
We would like to replace for vector spaces real numbers $\R$ with complex numbers $\bb C$
because of their obvious advantage that some characteristic polynomials have roots in $\bb C$
but not in $\R$.
In order to do so, we axiomatize all properties that our prototypes $\R$ and $\bb C$ satisfy.


\lb
\textbf{Definition (5.1.4)}
\lb
A \emph{field} is a set $F$ together with two operations called
\emph{addition} (+) and \emph{multiplication} $( \cdot )$ if the following axioms are satisfied:



\newpage
\lb
\textbf{Discussion}
\lb
Which of the following sets are fields? If they are not field,
explain one axiom that does not hold.
\begin{enumerate}
    \item[$\bb N$]
    ~\vspace{20pt}
    \item[$\bb Z$]
    ~\vspace{20pt}
    \item[$\bb Q$]
    ~\vspace{20pt}
    \item[$\bb R$]
    ~\vspace{20pt}
    \item[$\bb C$]
\end{enumerate}













\end{document}
