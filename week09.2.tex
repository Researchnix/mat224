\documentclass[letterpaper, 10pt]{article}
\usepackage[margin=1in]{geometry} 
\usepackage{amsmath,amsthm,amssymb,scrextend}
\usepackage{fancyhdr}
\pagestyle{fancy}
\usepackage{silence}
\WarningFilter{latex}{You have requested package}
\input{ltx/pkg/preamble}





\begin{document}

\lhead{MAT224 Linear Algebra II}
\chead{Complex numbers and Fields}
\rhead{Week 09 II}

\title{Linear Algebra II \\ \Large{MAT224}}
\author{Lennart Döppenschmitt}
% \maketitle
% \tableofcontents

\section*{Complex numbers}%
\textbf{Textbook:} Section 5.1



\lb
\textbf{Motivation}
\lb
We have seen that the matrix $ A = \begin{pmatrix} 0 & -1 \\ 1 & 0\end{pmatrix}$ has the
characteristic polynomial $c_A(λ) = λ^2 + 1$ which has no real roots.
A root would be a square root $\sqrt{-1}$. Complex numbers solve this problem.

\lb
\textbf{Definition}
\lb
The set of complex number $\bb C$ is the set vector space $\R^2$ with an additional
multiplication of two complex numbers
\[ \ttpl{a}{b} \ast \ttpl{c}{d} = \ttpl{ac-bd}{ad+bc} \]

\lb
\textbf{Remark}
\lb
One often writes these coordinate tuples as vectors in the basis $ \cb{1, i}$.
The coordinate vector $\ttpl{a}{b}$ corresponds then to $a + i b$.


\lb
\textbf{Discussion}
\lb
\begin{enumerate}
    \item Express $\ttpl{2}{3}$ and $\ttpl{-1}{4}$ in the basis and determine
        $\ttpl{2}{3} \ast \ttpl{-1}{4}$
    \item Let $W = \spn{1} \subseteq \bb C$. Show that $W$ is closed
        under the multiplication $\ast$.
    \item Can we say that $W$ is isomorphic to $\R$? If so, why?
\end{enumerate}





\newpage
\lb
\textbf{Definition}
\lb
Let $z = a + ib ∈ \bb C$. We call $Re(z) = a$ the \emph{real part} of $z$ and $Im(z) = b$
the \emph{imaginary part} of $z$.



\lb
\textbf{Discussion}
\lb
Let $p(x) = x^2 + 1$ and $q(x) = x^4 + 1$.
\begin{enumerate}
    \item Find all roots of $p(x)$ in $\bb C$.
    \item Find all solutions to the equation $i \cdot z = 1$. Can you now give meaning to
        the expression $\iv i$?
    \item Find all roots of $q(x)$ in $\bb C$.
    \item What is the inverse of a complex number $a + ib$ in general?
\end{enumerate}


\newpage
\lb
\textbf{Discussion}
\lb
Let $w = a+ ib$ and consider $\map{\bb C}[T_w]{\bb C}$ to be $T_w(z) = w \cdot z$.
\begin{enumerate}
    \item Is $T_w$ a linear transformation between real vector spaces? If it is, what is the
        matrix representing it?
    \item Under what condition is $T_w$ invertible?
\end{enumerate}


\newpage
\lb
\textbf{Discussion}
\lb
Let  $\map{\bb C}[T]{\bb C}$ to be the function $T(a + ib) = a - ib$.
\begin{enumerate}
    \item Is $T_w$ a linear transformation between real vector spaces? If it is, what is the
        matrix representing it?
    \item What are eigenvalues and eigenspaces of $T$?
    \item What is $T \circ T$? Is $T$ invertible?
\end{enumerate}














\newpage
\section*{Fields}%
\textbf{Textbook:} Section 5.1

\lb
\textbf{Goal}
\lb
We would like to replace for vector spaces real numbers $\R$ with complex numbers $\bb C$
because of their obvious advantage that some characteristic polynomials have roots in $\bb C$
but not in $\R$.
In order to do so, we axiomatize all properties that our prototypes $\R$ and $\bb C$ satisfy.


\lb
\textbf{Definition (5.1.4)}
\lb
A \emph{field} is a set $F$ together with two operations called
\emph{addition} (+) and \emph{multiplication} $( \cdot )$ if the following axioms are satisfied:



\newpage
\lb
\textbf{Discussion}
\lb
Which of the following sets are fields? If they are not field,
explain one axiom that does not hold.
\begin{enumerate}
    \item[$\bb N$]
    ~\vspace{20pt}
    \item[$\bb Z$]
    ~\vspace{20pt}
    \item[$\bb Q$]
    ~\vspace{20pt}
    \item[$\bb R$]
    ~\vspace{20pt}
    \item[$\bb \R^+$] $  = \cb{x \in \R ~ \vert ~ x \geq 0}$
    ~\vspace{20pt}
    \item[$\bb C$]
\end{enumerate}

\lb
\textbf{Example}
\lb
The finite field $\bb F_p$ with $p$ elements where $p$ is a prime.

\vspace{50pt}
\lb
\textbf{Exercise}
\lb
\begin{enumerate}
    \item Write out the addition and multiplication tables for the field $\bb F_3$.
    \item How can you tell that addition and multiplication are commutative?
    \item How can you conclude that $0$ is the additive identity?
    \item How can you conclude that every element has an additive inverse?
    \item How can you conclude that 1 is the multiplicative identity?
    \item How can you conclude that every non-zero element has a multiplicative inverse?
\end{enumerate}



\newpage
\lb
\textbf{Discussion}
\lb
\begin{enumerate}
    \item What are the roots of $p(x) = x^2 -1$ in $\bb F_2$? How about  $\bb F_3$?
    \item For each of the fields $\R, \Q, \bb C$ and $\bb F_2$, if possible give an example of
        \begin{enumerate}
            \item a polynomial that has a root in the field
            \item a polynomial that does not have a root in the field.
        \end{enumerate}
\end{enumerate}


\vspace{200pt}
\lb
\textbf{Definition (5.1.11)}
\lb
A field $F$ is called algebraically closed if every non-constant polynomial with coefficients in
$F$ has a root in $F$.

\vspace{20pt}
\lb
\textbf{Discussion}
\lb
Decide which of the fields $\R, \Q, \bb C$ and $\bb F_2$ is algebraically closed.



\vspace{200pt}
\lb
\textbf{Theorem (5.1.12 Fundamental Theorem of Algebra)}
\lb
$\bb C$ is algebraically closed, that is, every plynomial of degree $n$
\[ p(x) = a_n x^n + \dots + a_0 \]
has $n$ roots counted with multiplicity.










\newpage
\section*{Vector Spaces over a Field}%
\textbf{Textbook:} Section 5.2


\lb
\textbf{Definition 5.2.1}
\pr
A vector space $(V, + , \cdot)$ over a field $F$ consists of a set $V$ and two operations that we
call \emph{addition} $(+)$ and \emph{scalar multiplication} $\cdot$
\[ \map{V \times V}[+]{V} \]
\[ \map{F \times V}[\cdot]{V} \]
such that the following axioms hold
\begin{enumerate}
    \item
        (additive closure)
        \qquad $\vec x + \vec y ∈ V$,
        for all $\vec x, \vec y ∈ V$
    \item
        (multiplicative closure)
        \qquad $α \cdot \vec x ∈ V$,
        for all $\vec x ∈ V$ and scalars $α ∈ F$
    \item
        (commutativity)
        \qquad $\vec x + \vec y = \vec y + \vec x$,
        for all $\vec x, \vec y ∈ V$
    \item
        (additive associativity)
        \qquad $(\vec x + \vec y) + \vec z =\vec x + (\vec y + \vec z)$,
        for all $\vec x , \vec y , \vec z ∈ V$
    \item
        (additive identity)
        \qquad There exists a vector $\vec 0 ∈ V$ such that $\vec x + \vec 0 = \vec x$
        for all $\vec x ∈  V$
    \item
        (additive inverse)
        \qquad For each $\vec x ∈ V$, there exists a vector $- \vec x ∈ V$
        with the property that $\vec x + (- \vec x) = \vec 0$
    \item
        (multiplicative associativity)
        \qquad $(α \cdot β) \cdot \vec x = α \cdot ( β \cdot \vec x)$,
        for all $α, β ∈ F$ and $\vec x ∈ V$
    \item
        (distributivity over vector addition)
        \qquad $α \cdot ( \vec x + \vec y) = α \vec x + α \vec y$,
        for all $α ∈ F$ and $\vec x, \vec y ∈ V$
    \item
        (distributivity over scalar addition)
        \qquad $(α + β) \cdot  \vec x = α \vec x + β \vec x$,
        for all $α, β ∈ F$ and $\vec x ∈ V$
    \item
        (identity property)
        \qquad $1 \cdot \vec x = \vec x$,
        for all $\vec x ∈ V$, $1 ∈ F$
\end{enumerate}



\vspace{20pt}

\lb
\textbf{Examples}
\begin{enumerate}
    \item $F^n$, the set of $n$-tuples in a field $F$
    \item $\mat{n}{F}$, the set of $n\times n$ matrices with entries from $F$.
    \item $\pol{n}{F}$, the set of polynomials of degree $n$ with coefficients in $F$.
    \item $\mathbb{C}^n$ may be thought of as a complex vector space or a real vector space.
\end{enumerate}


\vspace{80pt}
\lb
\textbf{Remark}
\lb
Specifying the field for a vector space is important!
The vectors $\ttpl{i}{2}$ and $\ttpl{-1}{2i}$ in are dependent in $\mathcal{C}^2$ as a complex
vector space, but independent in $\mathcal{C}^2$ as a real vector space.

\newpage
\lb
\textbf{Discussion}
\lb
Consider $V = \R$ as a vector space over $\Q$.
\begin{enumerate}
    \item Is the family of vector $ \cb{1, \sqrt{2}}$ linearly independent or dependent?
    \item What is the dimension of the subspace $\spn{1, \sqrt{2}, \sqrt{3}, \sqrt{4}}$?
    \item What do the above results suggest about $\dim(V)$?
\end{enumerate}

\vspace{200pt}
\lb
\textbf{Remark}
\lb
Everything we did about vector spaces, matrices, inverse matrices, determinants, eigenvalues, ...
can be applied to a vector space over a field. See Example (5.2.8-10) in the book for examples.

\end{document}
