\documentclass[letterpaper, 10pt]{article}
\usepackage[margin=1in]{geometry} 
\usepackage{amsmath,amsthm,amssymb,scrextend}
\usepackage{fancyhdr}
\pagestyle{fancy}
\usepackage{silence}
\WarningFilter{latex}{You have requested package}
\input{ltx/pkg/preamble}





\begin{document}

\lhead{MAT224 Linear Algebra II}
\chead{Eigenvalues, Eigenvectors and Diagonalizability}
\rhead{Week 09 I}

\title{Linear Algebra II \\ \Large{MAT224}}
\author{Lennart Döppenschmitt}
% \maketitle
% \tableofcontents

\section*{Eigenvalues and Eigenvectors Pt II}%
\textbf{Textbook:} Section 4.1

\lb
\textbf{Proposition (4.1.5)}
\lb
A nonzero vector $\vec v ∈ V$ is an eigenvector of $T$ with eigenvalue $λ$ if and only if
\[ \vec v ∈ \ker(T - λ \cdot \tx{id}_V) \]
\begin{proof}
    
\end{proof}
\vspace{200pt}
\lb
\textbf{Discussion}
\lb
For a transformation $T ∈ \cal L(V)$ which of these statements are equivalent to 
\emph{$λ$ is an eigenvalue of $T$}?
\begin{enumerate}
    \item $T(\vec v) = λ \vec v$ for some $\vec v ∈ V$
    \item $\ker(T - λ \tx{id}_V) \neq \cb{0}$
    \item $T - λ \tx{id}_V$ is not an isomorphism
\end{enumerate}




\vspace{30pt}
\lb
\q{\textbf{Question}}
\lb
\q{So for which $λ ∈ \R$ can we expect to find eigenvectors?}







\newpage
\lb
\textbf{Proposition (4.1.9)}
\lb
$λ ∈ \R$ is an eigenvalue of $T ∈ \cal L(V)$ if and only if
\[ \det{T - λ \cdot \tx{id}_V} = 0 \]
\ble{That is, there are eigenvectors $\vec v ∈ V$ with eigenvalue $λ$.}
\begin{proof}
\end{proof}




\vspace{200pt}
\lb
\textbf{Proposition (4.1.6)}
\lb
For a given eigenvalue $λ$ of  $T ∈ \cal L(V)$ the set of all eigenvectors
\[ E_λ = \cb{\vec v ∈ V ~ \vert ~ T(\vec v) = λ  \vec v } \]
together with with the zero vector $\vec 0$
is a subspace of $V$ called the \emph{$λ$-eigenspace} of $T$.
\begin{proof}
\end{proof}






\newpage
\lb
\textbf{Example (4.1.1)}
\lb
Let's have another look at example (4.1.1) from the book.
Let $\map{\R^2}[T_A]{\R^2}$ be the linear transformation represented by the matrix
\[ \begin{pmatrix} 2 & 1 \\ 1 & 2 \end{pmatrix} \]
Compute its eigenvalues and the corresponding eigenspaces with the theory introduced above.




\newpage
\lb
\textbf{Definition}
\lb
Let $A$ be an $n \times n$ matrix. The polynomial $c_A(λ) = \det{A - λ I_n}$ is called the
\emph{characteristic polynomial} of A.

\vspace{30pt}
\lb
Following this definition and proposition 4.1.9, we can say that eigenvalues of a matrix $A$
will be the roots of its characteristic polynomial.



\vspace{170pt}
\lb
\textbf{Proposition (4.1.12)}
\lb
Similar matrices have equal characteristic polynomial
\begin{proof}
\end{proof}






\vspace{200pt}
\lb
\textbf{Corollary}
\lb
The characteristic polynomial for a transformation $T ∈ \cal (L)$ as
\[ c_T(λ) =  \det{[T]_α^α - λI_n} \]
does not depend on the choice of basis.







\newpage
\lb
\textbf{True or False}
\begin{enumerate}
    \item[$\square$]
    $T ∈ \mathcal{L}(V)$ is an isomorphism if and only if $0$ is not an eigenvalue.
    \item[$\square$]
    Every transformation has at least one eigenvalue.
    \item[$\square$]
    There are infinitely many eigenvectors to every eigenvalue of a transformation.
    \item[$\square$]
    There is at least one eigenvector to every eigenvalue of a transformation.
    \item[$\square$]
    An $n \times n$ matrix can have at most $n$ distinct eigenvalues.
\end{enumerate}

\vspace{100pt}
\lb
\textbf{Discussion}
\lb
Suppose the transformation $T ∈ \mathcal{L}(V)$ is represented by the matrix
\[ \begin{pmatrix}
    2 & 1 & 0 \\
    0 & 2 & 0 \\
    2 & 3 & 1
\end{pmatrix}
\]
with respect to some basis $α$.
\begin{enumerate}
    \item Find the characteristic polynomial $c_T(λ)$.
    \item Find all eigenvalues of $T$.
    \item Find a basis for each eigenspace of $T$.
\end{enumerate}



\newpage
\section*{Diagonalizability}%
\textbf{Textbook:} Section 4.2

\lb
\textbf{Definition (4.2.1)}
\lb
A linear transformation $ T ∈ \mathcal{L}(V)$ on a finite dimensional vector space $V$ is
said to be \emph{diagonalizable} if there exists a basis of $V$
consisting entirely of eigenvectors of $T$.

\lb
\ble{Why does this definition makes sense?}
Try to find the matrix $[T]_α^α$ in a basis of eigenvectors $α = \cb{\vec{α} _1, \ldots, \vec{α}_n}$


\vspace{200pt}
\lb
\textbf{Goal}
\lb
Find a condition to determine whether a transformation is diagonalizable or not.

\lb
\textbf{Definition}
\lb
Let $λ$ be an eigenvalue of a linear transformation $T$ on $V$.
\begin{enumerate}
    \item The \emph{algebraic multiplicity} $m_λ$ of $λ$ is the degree with which 
        $c_T$ vanishes at $λ$.
    \item The \emph{geometric multiplicity} of $λ$ is the dimension of the eigenspace $E_λ(T)$
\end{enumerate}



\newpage
\lb
\textbf{Proposition (4.2.4)}
\lb
Let $\vec v_1, \ldots \vec v_k$ be eigenvectors to distinct eigenvalues $λ_i$ of a linear transformation $\map{V}[T]{V}$,
then $ \cb{\vec v_1, \ldots, \vec v_n}$ is linearly independent in $V$.
\begin{proof}
\end{proof}




\newpage
\lb
\textbf{Corollary (4.2.5)}
\lb
Let $T$ be a linear transformation on $V$ with distinct eigenvectors $λ_1, \ldots, λ_k$ and for
each eigenvalue $λ_j$ consider a linearly independent family of eigenvectors
\[ \cb{\vec v_1^j, \ldots, \vec v_{n_j}^j} \]
in $E_{λ_j}$.
Then the union of all these families of eigenvectors 
\[ S = \cb{\vec v_1^j, \ldots, \vec v_{n_1}^j} \cup \ldots
\cup \cb{\vec v_1^k, \ldots, \vec v_{n_k}^k} \]
is linearly independent in $V$.
\begin{proof}
\end{proof}


\vspace{300pt}
\lb
\textbf{Discussion}
\lb
What is the intersection of eigenspaces to distinct eigenvalues?


\newpage
\lb
\textbf{Proposition (4.2.6)}
\lb
For every linear transformation $T$ on a finite dimensional vector space $V$ the geometric 
multiplicity is bounded by $1$ and the algebraic multiplicity.
\[ 1 \leq \dim{E_λ}(T) \leq m_λ \]
\begin{proof}
\end{proof}

\vspace{200pt}
\lb
\textbf{Theorem (4.2.7)}
\lb
For a linear transformation $T$ on a finite dimensional vector space $V$
with distinct eigenvalues $λ_1, \ldots, λ_k$.
Then $T$ is diagonalizable if and only if
\begin{enumerate}
    \item $m_{λ_1} + \cdots + m_{λ_k} = \dim{V}$
    \item for each $i$, $\dim{E_{λ_i}} = m_{λ_i}$
\end{enumerate}
\textbf{Remark:} The first condition can be dropped if we assume that all roots of $c_T(λ)$ has
are real valued. We will see a discussion later.
\begin{proof}
\end{proof}


\newpage
\lb
\textbf{Example (4.2.3)}
\lb
Diagonalize, if possible the transformation $T$ on $\R^3$ given by the matrix
\[ A = \begin{pmatrix}
    2 & 2 & 0 \\
    0 & 1 & 0 \\
    0 & 1 & 2
\end{pmatrix}
\]




\vspace{300pt}
\lb
\textbf{Discussion}
\lb
Can you argue with the transformation given by
\[ B = \begin{pmatrix}
    0 & -1 & 0 \\
    1 & 0 & 0 \\
    0 & 0 & 0
\end{pmatrix}
\]
why the condition $m_{λ_1} + \cdots + m_{λ_k} = \dim{V}$ is important in the above theorem?



\newpage
\lb
\textbf{Remarks}
\lb







\end{document}
