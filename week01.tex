\documentclass[letterpaper, 10pt]{article}
\usepackage[margin=1in]{geometry} 
\usepackage{amsmath,amsthm,amssymb,scrextend}
\usepackage{fancyhdr}
\pagestyle{fancy}
\usepackage{silence}
\WarningFilter{latex}{You have requested package}
\input{ltx/pkg/preamble}





\begin{document}

\lhead{MAT224 Linear Algebra II}
\chead{Vector spaces and subspaces}
\rhead{Week 01}

\title{Linear Algebra II \\ \Large{MAT224}}
\author{Lennart Döppenschmitt}
\maketitle
% \tableofcontents


\section*{Welcome,}%
my name is Lennart, I will be your instructor for Linear Algebra II this term.
I am a graduate student in the maths department and with my advisor I work on differential
geometry and the moduli space of Kähler metrics.
Apart from math (surprisingly similar though), I enjoy rock climbing.
\lb
You can reach me with questions about the course via \textbf{email: }
\emph{lennart at math dot toronto dot edu}
\pr
For math questions this course uses \emph{Piazza}.
\lb
My \textbf{office hours} take place on Wednesdays at 2 pm (EST) on Zoom.

\lb
\lb
\textbf{A few things about the class:}
\begin{itemize}
    \item
        Details can be found in the syllabus on Quercus
        \href{https://q.utoronto.ca}
        {\ble{\texttt{q.utoronto.ca}}}.


    \item The content will build on what you have learned in MAT223 with a
        particular focus on abstraction. Thinking abstractly is an important skill and
        makes arguments more powerful as it applies to a wider range of applications.

    \item
        A weekly class schedule, which we will follow closely, can be found at
        the end of the syllabus.

    \item
        All lectures will be recorded and made availble on YouTube for
        \emph{one} week, link will follow.
        Office hours will \emph{not} be recorded.
\end{itemize}





\lb
\textbf{About the lectures:}
\begin{itemize}
    \item
        Our lectures will be a mix of me explaining new math to you, time for you to try out
        what you just learned in short exercises and discussions and then a discussion round
        where we discuss the answer together.
    \item
        I will try to give Definitions, Theorems, etc. the same numbers as in the textbook.
    \item
        These notes will be available on Quercus.
        % Alternatively, you can access them on 
        % \pr
        % \href{https://www.github.com/researchnix/mat224}
        % {\ble{\texttt{github.com/researchnix/mat224}}}.
        % \pr
        % (The notes might be under construction up until the start of the lecture, but
        I encourage you to have a copy of them ready for the lecture to fill in
        the blanks and scribble down ideas for the discussions.
    \item
        Please let me know when you find mistakes in the notes, there will be plenty.
\end{itemize}

\lb
% \textbf{Miscellaneous:}
\textbf{About Zoom:}
\begin{itemize}
    \item We will use the same Zoom meeting ID for lectures and office hours.
            Please do \textbf{NOT} share the ID or password publically.
    \item To join you need a Zoom account with an official UofT email.
    \item Please mute yourself in Zoom.
    \item For questions, you can \emph{virtually} raise your hand and unmute yourself when
        I ask you to.
    \item Please only use the chat function for lecture relevant messages and please don't spoil
        answers to give everyone time to think about the questions.
\end{itemize}
\lb
\textbf{Thank you } to Professor Sean Uppal and Jeffrey Im for providing me with their old slides from a previous iteration of this course as a reference!


\newpage
\section*{Introduction}%
\label{sec:introduction}

Let us start by going over mathematical basics that we will need for this course.
\lb
\begin{itemize}
    \item
        A \emph{set} is a collection of objects. Examples include the set of integers
        $\Z$, the set of real numbers $\R$, the set of nonnegative integers $\Z_{\geq 0}$.

        \pr
        $A \subseteq B$ indicates that every element in $A$ is
        also an element in $B$. We say in this case that $A$ is a \emph{subset} of $B$.

        \pr
        A subset may be declared by pruning a set with a specified condition. For example, the
        set of even integers and the set of odd integers are
        \[ \bb E = \cb{ n ∈ \Z ~\vert~ n \tx{ is even }} \subseteq \Z\]
        \[ \bb O = \cb{ n ∈ \Z ~\vert~ n \tx{ is odd}} \subseteq \Z\]

        \pr
        If $A \subseteq B$ and at least one element of $B$ is not in $A$, we say that $A$ is
        properly contained in $B$, in symbols $A \subsetneq B$.

    \newpage
    \item
        A function $\map{A}[f]{B}$ between sets is an assignment that chooses for every $a ∈ A$
        an element $f(a) = b ∈ B$. For example,
        \begin{align*}
            &\map{\Z}[g]{\Z} \\
            &x \mapsto 2x + 1
        \end{align*}
        sends every real number $x$ to twice its value plus one. That is, $g(1) = 3$,
        $g(2) = 5$, and so on $\ldots$.

        \lb
        The notation $\map{A}[f]{B}$ moreover describes that we are allowed to apply f to
        elements $a ∈ A$, we therefore call $A$ the \emph{domain} of $f$.
        \lb
        The set $B$ is called the \emph{codomain} or \emph{target} of $f$, this is where $f$
        takes values in.
        \lb
        for $a ∈ A$ we call $f(a)$ the \emph{image of $a$ under $f$}.
        \lb
        The set of all possible functions between two sets $A$ and $B$ is
        denoted by $\mathcal{F}(A,B)$.

        \pr
        For any set $A$, there is the function $\map{A}[\id{A}]{A}$ that sends every element
        back to itself.
        \[ \id{A}(a) = a \qquad \tx{for all } a ∈ A \]

    \newpage
    \item
        A function $\map{A}[f]{B}$ can have the following properties:
        \begin{itemize}
            \item[] \emph{injective} \\
                No two distinct elements $a \neq b$ in $A$ have the same value
                $f(a) \neq f(b)$ under $f$. \\
                In other words, if $f$ is injective, then $f(a) = f(b)$ implies that
                $a = b$.
            \item[] \emph{surjective} \\
                Every element $b$ in $B$ is the image of some element $a$ in $A$. \\
                Equivalently, for all $b ∈ B$ there is an $a ∈ A$ such that $b = f(a)$.
            \item[] \emph{bijective}\\
                The function is both injective and surjective.
        \lb
        \q{Which of these properties apply to $g(x) = 2x + 1$ defined above as a function
        from $\Z$ to $\Z$?} \q{If not, how can you change the definition to make it bijective?}
        \end{itemize}

    \newpage
    \item
        Two functions of the form
        $\map{A}[f]{B}$ and
        $\map{B}[g]{C}$ may be concatinated. This means, whatever the first $f$ functions spits
        out is fed back into the next function $g$.
        \[ a \mapsto f(a) \mapsto g(f(a)) \]
        This is formally called \emph{composition} of $f$ and $g$ and denoted by
        \[ (g \circ f) (a) = g(f(a)) \]
        Notice, this only makes sense if the domain of $g$ is also the codomain of $f$.

    \newpage
    \item
        An \emph{inverse} of a function $\map{A}[f]{B}$ is a function in the \emph{opposite}
        direction $\map{B}[h]{A}$ with the property that
        \[ h(f(a)) = a \qquad \tx{for all } a ∈ A \]
        and
        \[ f(h(b)) = b \qquad \tx{for all } b ∈ B \]
        Intuitively, this means $h$ is undoing whatever $f$ was doing. For example,
        the function $h(y) = \frac{y-1}{2}$ from $\bb O$ to $\Z$ is an inverse to
        the function $g(x) = 2x + 1$.
\end{itemize}

\pr
We will wrap up this introduction with the following theorem

\lb
\textbf{Theorem}
\lb
A function $\map{A}[f]{B}$ is invertible if and only if it is bijective
% \begin{proof}
% \end{proof}









\newpage
\section*{Vector spaces}%
\label{sec:vector spaces}
\pr
\textbf{Textbook:} Section 1.1



\lb
\textbf{Definition 1.1.1}
\pr
A (real) vector space $(V, + , \cdot)$ consists of a set $V$ and two operations that we
call \emph{addition} $(+)$ and \emph{scalar multiplication} $\cdot$
\[ \map{V \times V}[+]{V} \]
\[ \map{\R \times V}[\cdot]{V} \]
such that the following axioms hold
\begin{enumerate}
    \item
        (additive closure)
        \qquad $\vec x + \vec y ∈ V$,
        for all $\vec x, \vec y ∈ V$
    \item
        (multiplicative closure)
        \qquad $α \cdot \vec x ∈ V$,
        for all $\vec x ∈ V$ and scalars $α ∈ \R$
    \item
        (commutativity)
        \qquad $\vec x + \vec y = \vec y + \vec x$,
        for all $\vec x, \vec y ∈ V$
    \item
        (additive associativity)
        \qquad $(\vec x + \vec y) + \vec z =\vec x + (\vec y + \vec z)$,
        for all $\vec x , \vec y , \vec z ∈ V$
    \item
        (additive identity)
        \qquad There exists a vector $\vec 0 ∈ V$ such that $\vec x + \vec 0 = \vec x$
        for all $\vec x ∈  V$
    \item
        (additive inverse)
        \qquad For each $\vec x ∈ V$, there exists a vector $- \vec x ∈ V$
        with the property that $\vec x + (- \vec x) = \vec 0$
    \item
        (multiplicative associativity)
        \qquad $(α \cdot β) \cdot \vec x = α \cdot ( β \cdot \vec x)$,
        for all $α, β ∈ \R$ and $\vec x ∈ V$
    \item
        (distributivity over vector addition)
        \qquad $α \cdot ( \vec x + \vec y) = α \vec x + α \vec y$,
        for all $α ∈ \R$ and $\vec x, \vec y ∈ V$
    \item
        (distributivity over scalar addition)
        \qquad $(α + β) \cdot  \vec x = α \vec x + β \vec x$,
        for all $α, β ∈ \R$ and $\vec x ∈ V$
    \item
        (identity property)
        \qquad $1 \cdot \vec x = \vec x$,
        for all $\vec x ∈ V$, $1 ∈ \R$
\end{enumerate}


\lb
\textbf{Remark}
\begin{itemize}
    \item
        Think of a vector space as set, only that you are able to do more with its elements
        (you can add two of them together and multiply them by a number).
        Plus, there are 10 `rules' you know your vector space obeys to.
    \item
        For elements in a vector space $V$, we write $\vec x, \vec y, \ldots∈ V$.
        The textbook writes ${\bf x, y, } \ldots ∈ V$.
    \item
        We often abbreviate $α \cdot \vec x$ with $α \vec x$.
    \item
        Elements in a vetcor space a called vectors. Be aware that anything can be a vector,
        even functions for example.
    \item
        We say that a vector space is \emph{real} if the scalars are real numbers. For now every vector space is real, we will only later allow the scalars to be \emph{complex numbers}
        and such.
\end{itemize}



\newpage
\pr
\textbf{Examples}
    \begin{enumerate}
        \item The real numbers $\R$ form a vector space with 'usual' addition $+$
            and multiplication $\cdot$.
        \item An n-tuple of real numbers can be written as
            $\vec v = (v_1, v_2, \ldots, v_n)$ where
            each $v_i ∈ \R$. The set of n-tuples is a vector space denoted by $\R^n$.
            \pr
            The operations addition and componentwise are performed componentwise.
            The additive identity is the zero tuple $\vec 0 = (0, \ldots, 0)$.
        \item
            The set $\mat{n, m}{\R}$ of $n \times m$ matrices with componentwise addition
            and scalar multiplication.
        \item
            The set of polynomials of degree at most $n$
            \[ \pol{n}{\R} = \cb{a_0 + a_1 x + a_2 x^2 + \cdots + a_n x^n \vert a_0, \ldots, a_n ∈ \R } \] is
            a vector space. \\
            Addition of two polynomials is applied to coefficientwise and the identity element $\vec 0$ is the
            polynomial that is constantly zero $p(x) = 0$.
            \pr
            Because every term in a polynomial looks like $a_i x^i$ for some value of $i$ between $0$ and $n$, we can abbreviate
            \[ \Sum[i=0][n] a_i x^i = a_0 + a_1 x + a_2 x^2 + \cdots + a_n x^n \]
        \item
            The set
            \[ \cal F (\R, \R) \]
            of functions from the real numbers to the real numbers is a vector space.
            \pr
            What might be the operations $+$ and $\cdot$ ?
    \end{enumerate}
\vspace{20pt}
\textbf{Intuition}
\lb
In many cases vectors may be represented with arrows becasue they, too, have a
direction and a magnitude. But be careful, every analogy has its limitations.




\newpage
\lb
\textbf{Discussion}
\begin{enumerate}
    \item[(I)]
        Is the set of 2-tuples of integers $\Z^2$ a real vector space?
        \pr \textbf{Hint: } To verify that something is a vector space, we need to check
        \emph{all} axioms in the definition. However, to prove the contrary, it is enough to
        disprove \emph{one single} axiom!




    \vspace{400pt}
    \item[(II)]
        Is the set $\pol{n}{\R}'$ of polynomials of \emph{exactly} degree $n$ a vector space?
\end{enumerate}







\newpage
\subsection*{Some Properties of vector spaces}%
\label{sub:Some Properties of vector spaces}

Whenever we introduce a new mathematical \emph{object}, such as a vector space,
we may not take anything for granted. In some ways, vectors behave like real numbers
(addition, scalar multiplication, zero element, additive inverse $\ldots$) but in many ways
they do not!\\
For example, for $3 ∈ \R$ we can write $ \frac{1}{3} $, but for a vector
$\vec v ∈ V$ we may not wirte $\frac{1}{\vec v}$. To be a good mathematican, it is very helpful
to be extremely pedantic!



\lb
\textbf{Theorem (Cancellation)}
\lb
Let $V$ be a vector space and $\vec u, \vec v, \vec w ∈ V$. If
\[ \vec u + \vec w = \vec v + \vec w \]
the
\[ \vec u = \vec v \]

\begin{proof}
    \begin{align*}
        \vec u
        &= \vec u + \vec 0 \\
        &= \vec u + (\vec w - \vec w) \\
        &= (\vec u + \vec w) - \vec w \\
        &= (\vec v + \vec w) - \vec w \\
        &= \vec v + (\vec w - \vec w) \\
        &= \vec v + \vec 0 \\
        &= \vec v
    \end{align*}
\end{proof}




\lb
\textbf{Proposition}
Let $V$ be a vector space and $\vec v \in V$, then
\[ 0 \vec v = \vec 0 \]
\q{Explain the difference between $0$ and $\vec 0$.}

\begin{proof}
    \begin{align*}
        \vec 0 + 0 \vec v
        &= 0 \vec v \\
        &= (0 + 0) \vec v \\
        &= 0 \vec v + 0 \vec v
    \end{align*}
    So by the cancellation theorem we can simplify
    \[ \vec 0 + 0 \vec v = 0 \vec v \]
    to
    \[ \vec 0 = 0 \vec v \]
\end{proof}



\lb
\textbf{Proposition}
Let $V$ be a vector space and $\vec v \in V$, then
\[ -1 \cdot \vec v = - \vec v \]
\begin{proof}
    \begin{align*}
        - \vec v
        &= - \vec v + \vec 0 \\
        &= - \vec v + 0 \vec v \\
        &= - \vec v + (1 + (-1)) \vec v \\
        &= - \vec v + (1 + (-1)) \vec v \\
        &= - \vec v + 1 \vec v + (-1) \cdot \vec v \\
        &= - \vec v + \vec v + (-1) \cdot  \vec v \\
        &= \vec 0 + (-1) \cdot  \vec v \\
        &= -1 \cdot \vec v
    \end{align*}
\end{proof}





\newpage
\lb
Notice that the symbol $+$ does \emph{not} necessarily refer to the standard addition, it
could be defined in a different way as we can observe in the following.
\lb
\textbf{Discussion}
\lb
    Let $V$ be the set of 2-tuples of real numbers $\ttpl{u_1}{u_2}$.
    \lb
    Define addition of 2-tuples as 
    \[ \ttpl{u_1}{u_2} \diamond \ttpl{v_1}{v_2}  = \ttpl{u_1 + v_1 + 1}{u_2 + v_2 + 1} \]
    \lb
    Let scalar multiplication be given by
    \[ α \star \ttpl{u_1}{u_2} = \ttpl{α u_1 + α - 1 }{β u_2 + β - 1} \]
\lb
Is there an additive identity? Are there inverses?
Is $(V, \diamond , \star )$ a vector space?
\lb
\textbf{Hint} Look at the propositions from the previous page.






\newpage
\section*{Subspaces}%
\label{sec:Subspaces}
\pr
\textbf{Textbook:} Section 1.2

\lb
\textbf{Definition}
\pr
A \emph{subspace} $U$ of a vector space $(V, +, \cdot)$ is a subset $U \subseteq V$ that
is a vector space in its own right (with the same addition and scalar multiplication of $V$).

\lb
This is the same idea as for subsets, only that the sets have been `upgraded' to vector spaces.

\lb
\textbf{Exmaples}
\begin{enumerate}
    \item
        As a first example, consider the vector spaces of polynomials $\pol{n}{\R}$ and of
        functions $ \mathcal{F}(\R, \R)$ we have already encountered. Now, every polynomial
        with real coefficients is automatically a function $\map{\R}[]{\R}$.

        \lb
        \q{Is this enough to be a subspace?}

        \lb
        We also need to check that the operations of addition and scalar multiplication of
        polynomials are the same when we consider polynomials to be functions.
    \vspace{200pt}
    \item
        Sine we can view a vector in $\R^2$, that is a tuple with 2 entries $\ttpl{v_1}{v_2}$,
        also as a vector in $\R^3$, namely a tuple with 3 entries,
        by adding the value $0$ at the last entry,
        \[ \begin{pmatrix} v_1 \\ v_2 \\ 0 \end{pmatrix} \]
        it is clear that $\R^2 \subseteq \R^3$ as sets.

        \lb
        As before, we need to check addition and scalar multiplication.
\end{enumerate}




\newpage
\lb
\textbf{Discussion}
\lb
Let $V$ be a real vector space.
\begin{enumerate}
    \item
        Can $U \subseteq V$ be a subspace if $U$ is empty (i.e. has no elements)?
    \item
        Suppose $U$ is a subset of $V$, which vector space axioms does $U$
        automatically inherit from $V$?
    \item
        Following the previous question, which axioms are left to check for $U$
        to be a vector space?
\end{enumerate}



\newpage
\lb
As we observed, we don't need to start from scratch to check all 10 axioms for subspaces.


\lb
\textbf{Theorem 1.2.8}
\lb
Let $V$ be a vector space with a subset $U$. Then $U \subseteq V$ is a subspace if and only if
\begin{enumerate}
    \item $U$ is nonempty
    \item $α \vec u + \vec v ∈ U$ for all $\vec u, \vec v ∈ U$ and $α ∈ \R$.
\end{enumerate}
\begin{proof}
\end{proof}





\newpage
\lb
We are now equipped with a very powerful tool to quickly test if a subset is a subspace.
\lb
\lb
\textbf{Examples / Discussion}
\lb
Decide which of the following subsets in vector spaces are subspaces.
\begin{enumerate}
    \item
        Continuous functions $C^1(\R, \R)$ contained in all functions $\mathcal{F}(\R,\R)$

    \lb
    \item
        Invertible $n \times n$ matrices $\cb{A ∈ \mat{n}{\R} \vert A \tx{ is invertible}}$
        contained in all matrices $\mat{n}{\R}$


    \lb
    \item
        The set $ \Big \{ \ttpl{x}{y} ∈ \R^2 ~ \vert ~ x + y = 1 \Big \} $ in $\R^2$

    \lb
    \item
        Nonnegative real numbers $\R_{\geq 0} = \cb{r ∈ \R ~ \vert ~ r \geq 0}$ in $\R$

    \lb
    \item
        The plane of vectors
        $ \Bigg \{ \tttpl{x}{y}{z} ∈ \R^3 ~ \Bigg \vert ~ y =0 \Bigg \}$ in $\R^3$

    \lb
    \item
        Only the zero element $ \cb{\vec 0}$ in $\R^2$

    \lb
    \item
        The space of \emph{even} and \emph{odd} functions respectively
        \[ \mathcal{F}(\R, \R)^{\tx{even}} = \cb{f ∈ \mathcal{F}(\R, \R) \s f(x) = f(-x)} \]
        \[ \mathcal{F}(\R, \R)^{\tx{odd}} = \cb{f ∈ \mathcal{F}(\R, \R) \s -f(x) = f(-x)} \]
        in all functions $\mathcal{F}(\R,\R)$
\end{enumerate}

\lb
\lb
\textbf{Hint :} Remember that the easiest way to disprove a \emph{general} statement is to give
a counterexample.



\newpage
\pr
\textbf{Theorem}
\lb
Let $U, W \subseteq V$ be subspaces in a vector space $V$, then
\begin{enumerate}
    \item $U \cap W$ is a subspace in $V$ \quad (Theorem 1.2.13)
    \item $U + W$ is a subspace in $V$
\end{enumerate}
\begin{proof}
    
\end{proof}







\end{document}
