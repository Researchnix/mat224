\documentclass[letterpaper, 10pt]{article}
\usepackage[margin=1in]{geometry} 
\usepackage{amsmath,amsthm,amssymb,scrextend}
\usepackage{fancyhdr}
\pagestyle{fancy}
\usepackage{silence}
\WarningFilter{latex}{You have requested package}
\input{ltx/pkg/preamble}





\begin{document}

\lhead{MAT224 Linear Algebra II}
\chead{Triangular and Nilpotent Matrices}
\rhead{Week 10}

\title{Linear Algebra II \\ \Large{MAT224}}
\author{Lennart Döppenschmitt}
% \maketitle
% \tableofcontents

\section*{Triangular Form}%
\textbf{Textbook:} Section 6.1

\pr
A diagonalizable matrix is in its \emph{normal form} when it is diagonal. And every
diagonalizable matrix can be brought to a diagonal form with a change of basis to
a \emph{canonical basis}.
\lb
Some matrices are not diagonalizable, what are their canonical forms and canonical bases?
The remaining sections 6.1-6.4 answer this quetsion step by step.

\lb
For this entire lecture, let $V$ be a finite dimensional vector space over a field $F$.

\lb
\textbf{Definition}
\lb
A matrix $A ∈ \mat{n}{F}$ is called \emph{upper triangular} if all entries below the diagonal
are 0. For example,
\[
    \begin{pmatrix}
        \ast & \ast & \ast \\
        0 & \ast & \ast \\
        0 & 0 & \ast \\
    \end{pmatrix}
\]

\lb
\textbf{Definition (6.1.2)}
\lb
Let $\map{V}[T]{V}$ be a linear transformation. A subspace $W \subseteq V$ is called
\emph{invariant} or \emph{stable} under $T$ if $T(W) \subseteq W$.

\lb
\textbf{Examples}
\begin{enumerate}
    \item Let $\map{F^3}[P_{xy}]{F^3}$ be the projection on the xy-plane. Then the xy-plane is
        an invariant subspace for it.
    \item $ \cb{0} $ and $V$ are always invariant subspaces for any transformation
        $T ∈ \mathcal{L}(V)$.
\end{enumerate}

\vspace{40pt}
\lb
\textbf{Discussion}
\lb
Let $\map{V}[T]{V}$ be a linear transformation such that in a basis
$β = \cb{\vec v_1, \vec v_2, \vec v_3}$
\[ [T]_β^β = \begin{pmatrix}  2 & 1 & -1 \\  & 3 & 1 \\  &  & 3 \end{pmatrix} \]

\begin{enumerate}
    \item Is $\spn{\vec v_1}$ invariant?
    \item Can you find three subspaces $W_1 \subset W_2 \subset W_3 $ which are all invariant?
\end{enumerate}


\newpage
\lb
\textbf{Proposition (6.1.4)}
\lb
A linear transformation $\map{V}[T]{V}$ is upper triangular in a basis $β$ if and only if for
each $i$, $1 \leq i \leq \dim(V)$
\[ W_i = \spn{\vec v_1, \ldots, \vec v_i} \]
is invariant.
\begin{proof}
\end{proof}
\vspace{200pt}

\lb
\textbf{Definition (6.1.5)}
\lb
A linear transformation $\map{V}[T]{V}$ is called triangulizable if there exists a basis $β$
such that $[T]_β^β$ is upper trianglar.

\lb
We skip a technical result from proposition 6.1.6 in the book and directly state
\lb
\textbf{Theorem (6.1.8)}
\lb
A linear transformation $\map{V}[T]{V}$ is triangulizable if and only if
the characterstic polynomial $c_T(λ)$ has $dim(V)$ roots (counted with multiplicity).

\vspace{100pt}


\lb
\textbf{Discussion}
\lb
Let $A ∈ \mat{n}{\bb C}$. Why is there always an upper trianglar matrix $B ∈ \mat{n}{\bb C}$
which is similar to $A$?

\newpage
\lb
\textbf{Notation}
\lb
If $p(x) = a_n x^k + \dots + a_0$ is a polynomial in $\pol{k}{F}$ and $A ∈ \mat{n}{F}$, we
define
\[ p(A) = a_n A^k + \dots + a_0 I_n \]


\lb
\textbf{Discussion}
\begin{enumerate}
    \item Suppose $A, B ∈ \mat{n}{F}$ are similar matrices such that
        $A = Q B \iv Q$ for an invertible matrix $Q ∈ \mat{n}{F}$.
        Show that
        \[ p(A) = Q p(B) \iv Q \]
    \item Suppose that
        \[ A = \begin{pmatrix}
            λ_1 & \ast & \cdots& \ast \\
                & λ_2 & & \vdots \\
                & & \ddots &  \ast \\
                & & & λ_n
        \end{pmatrix}
    \]
    is an upper triangular matrix. Compute $c_A(λ)$ and $c_A(A)$.
\end{enumerate}



\vspace{200pt}
\lb
\textbf{Theorem (6.1.12)}
\lb
Let $\map{V}[T]{V}$ be a linear transformation and assume that
$c_A(λ)$ has $dim{V}$ roots in $F$. Then $c_T(T) = 0$.
\begin{proof}
    \emph{( Sketch of proof )}
\end{proof}



\newpage
\section*{Nilpotent Normal Forms}%
\textbf{Textbook:} Section 6.2

\lb
The next kind of matrices we want to bring into a normal form are nilpotent matrices. Remember
that for the reminder of these notes $V$ is a finite dimensional vector space over a field $F$.

\lb
\textbf{Definition}
\lb
A linear transformation $\map{V}[N]{V}$ is called
\emph{nilpotent} if $N^n = 0$ for some $n \geq 1$.
The least $n$ such that $N^n = 0$ is called the \emph{index} of the nilpotent transformation.


\lb
\textbf{Discussion}
\begin{enumerate}
    \item Suppose $\map{V}[N]{V}$ is nilpotent, does $N$ always have an eigenvector?
    \item Suppose $\map{V}[N]{V}$ is nilpotent and $λ$ is an eigenvalue of $N$. What can $λ$ be?
    \item Suppose $\map{V}[T]{V}$ has only one distinct eigenvalue $λ=0$ of multiplicity
        $ m_λ = \dim(V)$. Is $T$ nilpotent?
\end{enumerate}



\vspace{100pt}
\lb
\textbf{Observation}
\lb
For a nilpotent transformation $\map{V}[N]{V}$ and $\vec v ∈ V$ nonzero
\[ N^k (\vec v) = 0\]
for $1 \leq k \leq n$, but not necessarily $n = k$!



\vspace{140pt}
\lb
\textbf{Deifnition (6.2.1)}
\lb
Let $\map{V}[N]{V}$ be a nilpotent transformation on $V$ and $\vec v ∈ V$
nonzero with $k$ as above.

\begin{enumerate}
    \item The set $α = \cb{N^{k-1}(\vec v), N^{k-2}(\vec v), \ldots, \vec v}$ is called the
        \emph{cycle} generated by $\vec v$.
        \pr $\vec v$ is called the \emph{inital vector} of this cycle.
    \item The subspace generated by this cycle $C(\vec v) = \spn{α}$ is called the
        \emph{cyclic subspace} generated by $\vec v$.
    \item We call $k$ the \emph{length} of the cycle.
\end{enumerate}


\newpage
\lb
\textbf{Discussion}
\begin{enumerate}
    \item Is $ \frac{d^2}{dx^2} $ nilpotent on $\pol{n}{F}$? What is the index?
    \item Can you find a polynomial $p ∈ \pol{n}{F}$ that generates a cycle of length 3?
\end{enumerate}


\lb
Look at Example (6.2.2) in the book for another example.


\vspace{200pt}
\lb
\textbf{Proposition (6.2.3)}
\lb
Let $\map{V}[N]{V}$ be a nilpotent transformation on $V$ and $\vec v ∈ V$.
\begin{enumerate}
    \item If $\vec v$ generates a cycle of length $k$, then $N^{k-1}(\vec v)$ is
        an eigenvector of $N$ with eigenvalue $λ=0$.
    \item $C(\vec v)$ is an invariant subspace for $N$.
    \item The cycle $α$ generated by $\vec v$ is independent and hence a basis for $C(\vec v)$.
\end{enumerate}
\begin{proof}
\end{proof}


\newpage
\lb
\textbf{Notation (Cycle tableau)}
\lb
Let $\map{V}[N]{V}$ be a nilpotent transformation on $V$.
\begin{enumerate}
    \item Write for a cycle $α$ of length $k$ a row of $k$
        boxes to represent every element in $α$.
        \vspace{100pt}
    \item If we consider $r$ cycles $α_1, \ldots, α_r$ generated by $\vec v_1, \ldots, \vec v_r$
        each of length $k_1, \ldots, k_r$ write $r$ rows of boxes sorted by length.
        \vspace{100pt}
        \lb
        We call this the \emph{cycle tableau} of the cycles $α_1, \ldots α_r$.
\end{enumerate}


\newpage
\lb
\textbf{Discussion}
\lb
Consider the following cycle tableau of the cycles $α_1, α_2, α_3$ of a nilpotent transformation
$\map{V}[N]{V}$.
\vspace{100pt}
\lb
Which of the boxes necessarily correspond to elements in
\begin{enumerate}
    \item ker($N$)
    \item im($N$)
    \item im($N^2$)
    \item ker($N) \cap \tx{im}(N^3)$
\end{enumerate}




\newpage
\lb
\textbf{Discussion}
\lb
Let $\map{V}[N]{V}$ be a nilpotent transformation on a vector space $V$ of dimension $6$.
\begin{enumerate}
    \item If $\vec v$ generates a cycle $α$ of length $6$,
        \begin{enumerate}
            \item is $α$ a basis for $V$?
            \item What is $[N]_α^α$?
        \end{enumerate}
    \item If $α_1, α_2, α_3$ are cycles of lengths $1, 2$ and $3$ such that
        $β = α_1 \cup α_2 \cup α_3$ is linearly independent,
        \begin{enumerate}
            \item is $β$ a basis for $V$?
            \item What is $[N]_β^β$?
        \end{enumerate}

    \item
        Does the matrix $[N]_β^β$ depend on the particular elements in $β$?
\end{enumerate}

\newpage
\lb
\textbf{Notation}
\lb
\begin{enumerate}
    \item \emph{nilpotent Jordan block} of size $k$
    \vspace{150pt}
    \item direct sum of matrices
    \vspace{150pt}
    \item \emph{nilpotent Jordan matrix}
\end{enumerate}



\vspace{150pt}
\lb
\textbf{Goal:} For a nilpotent transformation, we want to find a basis consisting of cycles.
\lb
\q{How many cycles do we need? Which cycles can we use? How do we find these cycles?}


\newpage
\lb
\textbf{Proposition (6.2.4)}
\lb
Let $α_1, \ldots, α_r$ be cycles of lengths $k_i$ respectively, generated by $\vec v_i$.
If the set of eigenvectors
\[ \cb{N^{k_1 -1} (\vec v_1) , \ldots, N^{k_r -1} (\vec v_r)} \]
is linearly independent, then the union 
\[ α_1 \cup \cdots \cup α_r \]
is linearly independent.
\begin{proof}
\end{proof}



\vspace{300pt}
\lb
\textbf{Definition (6.2.5 \& 6.2.7)}
\lb
Let $\map{V}[N]{V}$ be a nilpotent transformation.
\begin{enumerate}
    \item Cycles $α_1, \ldots, α_r$ such that $ α_1 \cup \cdots \cup α_r $ is linearly independent are called \emph{non-overlapping} cycles.
    \item A basis of $V$ consisting of non-overlapping cycles for $N$ is called
        a \emph{canonical basis} for $N$.
\end{enumerate}


\newpage
\lb
\textbf{Theorem (6.2.8)}
\lb
Every nilpotent transformation $\map{V}[N]{V}$, has a canonical basis on $V$.



\vspace{40pt}
\lb
\textbf{Lemma (6.2.9)}
\lb
The cycle tableau of a canonical basis for $N$ has
\[ \dim(\ker(N^j)) - \dim(\ker(N^{j-1})) \]
boxes in column $j$.

\vspace{40pt}
\lb
\textbf{Discussion (6.2.10)}
\lb
Let $\map{V}[N]{V}$ be a nilpotent transformation such that
\begin{itemize}
    \item $\dim(\ker(N)) = 3$
    \item $\dim(\ker(N^2)) = 5$
    \item $\dim(\ker(N^3)) = 7$
    \item $\dim(\ker(N^4)) = 8$
\end{itemize}
What shape does the cycle tableau of a canonical basis for $N$ must have?
What is the canonical form of the transformation $N$ is such a basis?

\vspace{200pt}
\lb
\textbf{Corollary (6.2.11)}
\lb
The canonical form of a nilpotent transformation is unique up to reordering
the nilpotent Jordan blocks. ( By convention we sort them by size. )





\vspace{40pt}
\lb
\textbf{Next time: } How can we find such a canonical basis explicitly?







\end{document}
